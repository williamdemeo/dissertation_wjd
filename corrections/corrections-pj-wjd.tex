Corrections PJ

(to printed version you gave me)

p.iv 2nd par. "Thus the problem problem"   

[wjd: fixed]

p.5 2nd par. I may be pedantic here, but you are using two notations for tuples: (x,y)  (in the definition of kernel) and \langle a_0,a_1\langle on p.7 2nd par. Since the latter notation is also used consistently for \langle A,F\rangle, you could change all tuples to this notation, but there are many occurrences of either notation (p7. (x_i,y_i), p11. (g,g), p19. (f,g), (x,y), (x,y), (x_0,x_1...,x_{n-1}), (s,s,...,s), ...) so you could also just add a remark that you will use the two notations interchangeably (or ignore my being pedantic).

[wjd: fixed.  thanks, I meant to use (x,y) for all elements of algebras and reserve angled brackets for algebras.]

p.8 generated by all of them together -> generated by the union of them

[wjd: fixed]

p.9, line 4 "embeds every lattice in a countably infinite partition lattice"
is not true. Leave out "countably infinite" and then add a remark that for finite lattices Whitman's argument gives embeddings into countably infinite partition lattices.

[wjd: I think I should fix this by writing ``embeds every finite lattice in a countably infinite...'' Yes?]

p.10 "and subgroup lattices of cyclic groups are self-dual". This may be obvious, but is there a standard reference that can be given? Is it also true for abelian groups?

===> TODO:
[wjd: I don't recall if it's obvious, and I can't remember why it's true.  I will either think of the proof and insert a footnote, or else insert the reference.]

p.17, 1st par. "the the"

[wjd: fixed]

p.19, line 3 was in proved -> was proved in

[wjd: fixed.  Also I'm changing ``we will find that the proof is actually quite easy.'' I've changed my mind about this.]

p.19, idempotent and decreasing: (me being pedantic again - can ignore this) usually decreasing for functions means x < y ==> f(x) > f(y). The property you use is f(x)\le x for all x, which means f is pointwise below the identity function.

[wjd: right, good catch.  I've inserted a footnote explaining that I use decreasing to mean non-increasing.]

p.21, line 2 \lambda(L) is only defined on p.26.

[wjd: fixed]

p.21, line 2 \theta is not below \alpha -> \alpha is not below \theta

[wjd: fixed. good catch!]

p.21. "violated by some function" is also only defined on p.26.

===> TODO:
[wjd: Yes, I'll have to fix this]

p.21. I don't see why L_0 \cong 2 would be a problem for this proof. Does this case really need to be treated separately? 

Also, why worry about h being constant or the identity? By construction a\ne b, so h violates theta which is a partition in L, hence h is not constant or the identity.

Thm 2.4.1 Figure number is missing

[wjd: fixed]

p.25 3rd par. relations of a finite set -> relations on a finite set

[wjd: fixed]

p.30 is the notation \alpha^{\downarrow} taken from somewhere else? I've usually seen this as {\downarrow}\alpha (similarly for \uparrow).

===> TODO:
[wjd: I'll say what it means in a footnote, and I don't mind changing it.]

p.30 line -3, B\lessnotequal X -> B \subsetne X  (I'm not sure if my latex command names are exactly the correct ones, so please check)

[wjd: fixed]
